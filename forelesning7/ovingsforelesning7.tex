\documentclass[screen, aspectratio=169]{beamer}
\usepackage[T1]{fontenc}
\usepackage[utf8]{inputenc}
\usepackage{tikz, environ}
\usepackage{listings}

% Use the NTNU-temaet for beamer 
% \usetheme[style=ntnu|simple|vertical|horizontal, 
%     language=bm|nn|en, 
%     smalltitle, 
%     city=all|trondheim|alesund|gjovik]{ntnu2017}
\usetheme[style=horizontal,language=bm]{ntnu2015}

\usepackage[norsk]{babel}

\RequirePackage{listings, color, textcomp}
\lstset{
	tabsize=2,
	rulecolor=,
	basicstyle=\ttfamily\small,
	upquote=true,
	aboveskip={1.5\baselineskip},
	columns=fixed,
	showstringspaces=false,
	extendedchars=true,
	literate={æ}{{\ae}}1
			 {ø}{{{\o}}}1
			 {å}{{\aa}}1
			 {Æ}{{\AE}}1
			 {Ø}{{\O}}1
			 {Å}{{\AA}}1,
	breaklines=true,
	breakatwhitespace=true,
	escapeinside={(*}{*)},
	showtabs=false,
	showspaces=false,
	keepspaces=true,
	showstringspaces=false,
	frame=l,
	identifierstyle=\ttfamily,
	keywordstyle=\color[rgb]{1.0,0,0},
	keywordstyle=[1]\color[rgb]{0,0,0.75},
	%keywordstyle=[2]\color[rgb]{0.5,0.0,0.0},
	keywordstyle=[3]\color[rgb]{0.127,0.427,0.514},
	keywordstyle=[4]\color[rgb]{0.4,0.4,0.4},
	commentstyle=\color[rgb]{0.133,0.545,0.133},
	stringstyle=\color[rgb]{0.639,0.082,0.082},
	mathescape
}
\lstset{language=Python}

\title[Short title]{Øvingsforelesning 7 i Python (TDT4110)}
\subtitle{Lister, Strenger, Funksjoner}
\author[Ole-Magnus Pedersen]{Ole-Magnus Pedersen}
\institute[NTNU]{}
\date{}
%\date{} % To have an empty date

\NewEnviron{transparent}{
	\tikz\node[opacity=0.2,align=left,inner xsep=0]{\parbox[t]{\linewidth}{
			\BODY
		}};
	}

\hypersetup{
	colorlinks,
	urlcolor={blue!70!black}
}

\begin{document}

\begin{frame}
  \titlepage
\end{frame}

% Alternatively, special title page command to get a different background
% \ntnutitlepage

\begin{frame}{Oversikt}
	\begin{itemize}
		\item Praktisk Info
		\begin{transparent}
			\item Gjennomgang av Øving 6
			\item Programmering til Øving 7
		\end{transparent}
	\end{itemize}
\end{frame}

\begin{frame}{Praktisk info}
	\begin{itemize}
		
		\item Kollokviegrupper
		\begin{itemize}
			\item Snakk med studassen deres dersom dere føler dere har behov for kollokvie. 
			\item Studassene velger ut hvem som har behov for det og kan melde de seg på
			\item Foreløbig er kollokviene for Python onsdag kvelder
		\end{itemize}
		\item Auditorieøving 2
		\begin{itemize}
		    \item Må ikke tas av de som gjorde den 1., men teller som en øving
		    \item Er i Uke 44
		\end{itemize}
	\end{itemize}
\end{frame}



\begin{frame}{Oversikt}
	\begin{itemize}
		\begin{transparent}
			\item Praktisk Info
		\end{transparent}
		\item Gjennomgang av Øving 6
		\begin{transparent}
			\item Programmering til Øving 7
		\end{transparent}
	\end{itemize}
\end{frame}

\begin{frame}{Gjennomgang av Øving 6}
	\begin{itemize}
		\item 
	\end{itemize}
\end{frame}

\begin{frame}{Oversikt}
	\begin{itemize}
		\begin{transparent}
			\item Praktisk Info
			\item Gjennomgang av Øving 6
		\end{transparent}
		\item Programmering til Øving 7
	\end{itemize}
\end{frame}

\begin{frame}[fragile]{Repetisjon om lister}
	\begin{columns}
		\begin{column}{.4\textwidth}
			\begin{itemize}
				\item Har en del innebygde funksjoner
				\begin{itemize}
					\item Legge til - append/insert
					\item Fjerne .pop()
					\item Sortere - .sort()
					\item Reverse - .reverse() 
				\end{itemize}
				\item Kan sjekke om noe er i en liste%
				\begin{itemize}
				    \item \lstinline|if "a" in liste|
				\end{itemize}
				\item Kan iterere gjennom liste
			
			\end{itemize}
		\end{column}
		\begin{column}{.55\textwidth}
			\begin{lstlisting}
liste = [2, 3, 4]
liste.append(5) #[2, 3, 4, 5]
liste.insert(0, 1) #[1, 2, 3, 4, 5]
liste.pop() #[1, 2, 3, 4]

4 in liste #True
6 in Liste #False

for i in range(len(liste)):
    element = liste[i]
for element in liste:
			\end{lstlisting}
		\end{column}
	\end{columns}
\end{frame}

\begin{frame}[fragile]{Strenger}
	\begin{columns}
		\begin{column}{.6\textwidth}
			\begin{itemize}
				\item Strenger kan ses på som en samling av tegn
				\begin{itemize}
					\item Kan hente ut et tallverdi til tegn med å bruke ord("a")
					\item Kan få tilbake tegn-verdien ved å bruke chr(4)
				\end{itemize}
				\item Kan gjøre om streng til liste $\rightarrow$ list(streng)
				\item Kan gjøre om liste til streng $\rightarrow$ "".join(liste)
			\end{itemize}
		\end{column}
		\begin{column}{.4\textwidth}
			\begin{lstlisting}
bokstav = "a"
ord(bokstav)  #97
chr(97) #"a"

navn = "Vegard"
liste = list(navn)  #["v", "e", "g", "a", "r", "d"]
"".join(liste) #"Vegard"
			\end{lstlisting}
		\end{column}
	\end{columns}
\end{frame}

\begin{frame}{Oppgave 1 - Strenger}
	\begin{itemize}
		\item<+-> Finn desimal-verdien til bokstaven ‘a’, så finn bokstaven som har desimal-verdi 122
		\begin{itemize}
		    \item Hint: ord(), chr()
		\end{itemize}
		\item<+-> Skriv en funksjon finner antall plasser mellom to bokstaver i alfabetet og returnerer verdien (f og h = 2), (a og e = 4)
        	\begin{itemize}
		    \item Hint: ord(), return
		\end{itemize}
		\item<+-> Skriv en funksjon som finner forskjellen mellom alle bokstavene i to like lange ord
        	\begin{itemize}
		    \item Hint: bruk difference fra forrige oppgave
		\end{itemize}
		\item<+-> Skriv en funksjon som tar inn et ord, sorterer bokstavene i ordet i alfabetisk rekkefølge, og printer det nye ordet

		\item<+-> Lag en funksjon som bytter alle vokalene i et ord med en ny tilfeldig vokal
			\begin{itemize}
		    \item Hint: 	def funnyWord(word): , vokaler = [‘a’,’e’,’i’,’o’,’u’] ,
	for char in word: , if char in vokaler:

		\end{itemize}

	\end{itemize}
\end{frame}

\begin{frame}{Oppgave 2: Yatzy - Utvidelse}
	\begin{itemize}
		\item<+-> Lag en funksjon som lager en liste med 5 tilfeldige heltall mellom 1 og 6 - (Gjorde vi sist)
	
		\item<+-> Lag en funksjon som tar inn en liste og en index som parameter og endrer tallet på indexen til et nytt tilfeldig tall, så returnerer listen

		\item<+-> Utvid funksjonen til å ta inn en liste med terninger og en \textbf{liste} med indexer og triller alle terningene på de gitte indexene på nytt

		\item<+->  Lag en funksjon som spør spilleren hvilke terninger han/hun vil trille om igjen via indexer, og returnerer en liste med indexene, det skal se ut som dette:
		\begin{itemize}
    \item   Hvilke terninger vil du kaste på ny? (separer med komma uten mellomrom – 1,3,5
    \end{itemize}
    \item<+-> Lag en funksjon fullfører alle tre kastene i et yatzy-kast og som spør hva som skal kastes på ny etter kast en og to
    \begin{itemize}
        \item Hint: indexer = nytt\_kast\_indexer()
	mitt\_kast = nytt\_kast(mitt\_kast,indexer)
	def kast():

    \end{itemize}

	\end{itemize}
\end{frame}



\begin{frame}{Spørsmål}
	\begin{itemize}
		\item Spørsmål kan også sendes på mail til \href{mailto::vegahel@stud.ntnu.no}{vegahel@stud.ntnu.no}
	\end{itemize}
\end{frame}

\end{document}
