\documentclass[screen, aspectratio=169]{beamer}
\usepackage[T1]{fontenc}
\usepackage[utf8]{inputenc}
\usepackage{tikz, environ}
\usepackage{listings}

% Use the NTNU-temaet for beamer 
% \usetheme[style=ntnu|simple|vertical|horizontal, 
%     language=bm|nn|en, 
%     smalltitle, 
%     city=all|trondheim|alesund|gjovik]{ntnu2017}
\usetheme[style=horizontal,language=bm]{ntnu2015}

\usepackage[norsk]{babel}

\RequirePackage{listings, color, textcomp}
\lstset{
	tabsize=2,
	rulecolor=,
	basicstyle=\ttfamily\small,
	upquote=true,
	aboveskip={1.5\baselineskip},
	columns=fixed,
	showstringspaces=false,
	extendedchars=true,
	literate={æ}{{\ae}}1
			 {ø}{{{\o}}}1
			 {å}{{\aa}}1
			 {Æ}{{\AE}}1
			 {Ø}{{\O}}1
			 {Å}{{\AA}}1,
	breaklines=true,
	breakatwhitespace=true,
	escapeinside={(*}{*)},
	showtabs=false,
	showspaces=false,
	keepspaces=true,
	showstringspaces=false,
	frame=l,
	identifierstyle=\ttfamily,
	keywordstyle=\color[rgb]{1.0,0,0},
	keywordstyle=[1]\color[rgb]{0,0,0.75},
	%keywordstyle=[2]\color[rgb]{0.5,0.0,0.0},
	keywordstyle=[3]\color[rgb]{0.127,0.427,0.514},
	keywordstyle=[4]\color[rgb]{0.4,0.4,0.4},
	commentstyle=\color[rgb]{0.133,0.545,0.133},
	stringstyle=\color[rgb]{0.639,0.082,0.082},
	mathescape
}
\lstset{language=Python}

\title[Short title]{Øvingsforelesning 9 i Python (TDT4110)}
\subtitle{Dictionaries, Exception, Filhåndtering}
\author[Ole-Magnus Pedersen]{Ole-Magnus Pedersen}
\institute[NTNU]{}
\date{}
%\date{} % To have an empty date

\NewEnviron{transparent}{
	\tikz\node[opacity=0.2,align=left,inner xsep=0]{\parbox[t]{\linewidth}{
			\BODY
		}};
	}

\hypersetup{
	colorlinks,
	urlcolor={blue!70!black}
}

\begin{document}

\begin{frame}
  \titlepage
\end{frame}

% Alternatively, special title page command to get a different background
% \ntnutitlepage

\begin{frame}{Oversikt}
	\begin{itemize}
		\item Praktisk Info
		\begin{transparent}
			\item Gjennomgang av Øving 7
			\item Programmering til øving 9
		\end{transparent}
	\end{itemize}
\end{frame}

\begin{frame}{Oversikt}
	\begin{itemize}
		\begin{transparent}
			\item Praktisk Info
		\end{transparent}
		\item Gjennomgang av Øving 7
		\begin{transparent}
			\item Programmering til Øving 9
		\end{transparent}
	\end{itemize}
\end{frame}



\begin{frame}{Oversikt}
	\begin{itemize}
		\begin{transparent}
			\item Praktisk Info
			\item Gjennomgang av Øving 7
		\end{transparent}
		\item Programmering til Øving 9
	\end{itemize}
\end{frame}

\begin{frame}[fragile]{Dictionaries}
	\begin{columns}
		\begin{column}{.5\textwidth}
			\begin{itemize}
				\item ''Liste'' med selvvalgte indekser
				\begin{itemize}
					\item Kan sette inn nye verdier
					\item Hente ut eksisterende
					\item Verdier kan ha alle typer variabler
					\item få tak i nøkler. dict.keys()
					\item få tak i alt dic.items() 
					\item Legge til en annen - dict.update(dict2)
				\end{itemize}
				\item Sjekke om en key er i dictionary
				\begin{itemize}
				    \item if ''a'' in dict.keys()
				\end{itemize}
				\item Kan iterere gjennom alle elementer
			\end{itemize}
		\end{column}
		\begin{column}{.5\textwidth}
			\begin{lstlisting}
dictionary = {}
dictionary[''TDT4110''] = ''IT Grunnkurs''
dictionary[''TDT4105''] #KeyError

''TDT4110'' in dictionary.keys() #true

for key, value in dictionary.items(): 
    #Key, er alle nøklene, 
    #Value er alle verdiene
	print(key, '':'', value) 
	# TDT4110 : IT Grunnkurs
			\end{lstlisting}
		\end{column}
	\end{columns}
\end{frame}

\begin{frame}[fragile]{Filhåndtering}
	\begin{columns}
		\begin{column}{.4\textwidth}
			\begin{itemize}
				\item Må lukke filen etter å ha lest, with-funksjonen gjør dette enklere
				\item Filoperasjoner, ''r'', ''w'', ''a''
				\begin{itemize}
					\item Kan bruke + for å lage nye filer
				\end{itemize}
				\item Det du ønsker å gjøre
				\begin{itemize}
				    \item Lese inn filen
				    \item For hver linje i filen
				    \item Preprosesser (Fjern linjeskift etc)
				    \item Legg det i en liste? 
				\end{itemize}
			\end{itemize}
		\end{column}
		\begin{column}{.6\textwidth}
			\begin{lstlisting}
linjer = []
with open(''fil.txt'', ''r'') as file:

#Åpner filen med variabelen file
    for linje in file.readlines():
#Linje er nå hver linje i filen
        linje = linje.strip() #Fjerner ''whitespace'' på starten og slutten
        linje = linje.split() #Splitter ved alle mellomrom i fila (Hvis kun en verdi, per linje 
        linjer.append(linje) #Matrise med alle verdiene
			\end{lstlisting}
		\end{column}
	\end{columns}
\end{frame}

\begin{frame}{Oppgaver - Dictionaries}
	\begin{itemize}
		\item<+-> Lag en dictionary: Prices, med nøklene ''banana'' =10, ''apple''=  5, ''grape''=  15, ''orange''=25 
		
		\item<+-> Lag en funksjon, som tar inn en dictionary, en nøkkel, og en verdi, og oppdaterer dictionarien med den nye verdien
        \item<+-> Lag en funksjon som itererer gjennom dictionarien og printer ut hver nøkkel og verdi sammen med hverandre
		\item<+-> Skriv inn en funksjon, som tar inn en dictionary, og et tall - og returnerer alle nøklene i verdien, som du har råd til å kjøpe
        
		\item<+-> Lag en funksjon, som tar inn to dictionaries: En pris, og en ''handlekurv'' som ser slik ut. {''apple'': 2}. For alle tingene du har i handlekurven, finn ut total pris på alle varene dine, ved hjelp av pris-dictionaries
		

		\item<+-> Utvid funksjonen til å også ta inn en lager dictionary, med antall varer på lager. Funksjonen skal sjekke om det er nok varer på lager, kjøpe så mange som du har sagt i handlekurven, og redusere varelageret med så mange du kjøper

	\end{itemize}
\end{frame}

\begin{frame}{Gjennomgang - Filhåndtering}
	\begin{itemize}
		\item<+-> Skriv en funksjon som tar inn et filnavn og to tall, a og b. Programmet skal skrive til fil 100 tilfeldige tall mellom a  b, med linjeskift mellom. 
		\begin{itemize}
		    \item Hint: file.write(''\textbackslash n'')
		\end{itemize}
		\item<+-> Skriv en funksjon som legger til 20 nye tilfeldige tall i en fil
		\item<+-> Skriv en funksjon som tar inn overnevnte fil, og putter alle tallene i en liste, og returnerer lista.
		\item<+-> Skriv en funksjon som finner det største tallet i en slik fil. 
		\item<+-> \textbf{Demo}: Hva skjer hvis tallene er separert 5 og 5 tall på hver linje
	\end{itemize}
\end{frame}



\begin{frame}{Spørsmål}
	\begin{itemize}
		\item Spørsmål kan også sendes på mail til \href{mailto::vegahel@stud.ntnu.no}{vegahel@stud.ntnu.no}
	\end{itemize}
\end{frame}

\end{document}
