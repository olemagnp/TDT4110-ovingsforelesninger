\documentclass[screen, aspectratio=169]{beamer}
\usepackage[T1]{fontenc}
\usepackage[utf8]{inputenc}
\usepackage{tikz, environ}
\usepackage{listings}

% Use the NTNU-temaet for beamer 
% \usetheme[style=ntnu|simple|vertical|horizontal, 
%     language=bm|nn|en, 
%     smalltitle, 
%     city=all|trondheim|alesund|gjovik]{ntnu2017}
\usetheme[style=horizontal,language=bm]{ntnu2015}

\usepackage[norsk]{babel}

\RequirePackage{listings, color, textcomp}
\lstset{
	tabsize=2,
	rulecolor=,
	basicstyle=\ttfamily\small,
	upquote=true,
	aboveskip={1.5\baselineskip},
	columns=fixed,
	showstringspaces=false,
	extendedchars=true,
	literate={æ}{{\ae}}1
			 {ø}{{{\o}}}1
			 {å}{{\aa}}1
			 {Æ}{{\AE}}1
			 {Ø}{{\O}}1
			 {Å}{{\AA}}1,
	breaklines=true,
	breakatwhitespace=true,
	escapeinside={(*}{*)},
	showtabs=false,
	showspaces=false,
	keepspaces=true,
	showstringspaces=false,
	frame=l,
	identifierstyle=\ttfamily,
	keywordstyle=\color[rgb]{1.0,0,0},
	keywordstyle=[1]\color[rgb]{0,0,0.75},
	%keywordstyle=[2]\color[rgb]{0.5,0.0,0.0},
	keywordstyle=[3]\color[rgb]{0.127,0.427,0.514},
	keywordstyle=[4]\color[rgb]{0.4,0.4,0.4},
	commentstyle=\color[rgb]{0.133,0.545,0.133},
	stringstyle=\color[rgb]{0.639,0.082,0.082},
	mathescape
}
\lstset{language=Python}

\title[Short title]{Øvingsforelesning 10 i Python (TDT4110)}
\subtitle{Rekursjon og repetisjon}
\author[O.M. Pedersen]{Ole-Magnus Pedersen}
\institute[NTNU]{}
\date{}
%\date{} % To have an empty date

\NewEnviron{transparent}{
	\tikz\node[opacity=0.2,align=left,inner xsep=0]{\parbox[t]{\linewidth}{
			\BODY
		}};
	}

\hypersetup{colorlinks,
	urlcolor={blue!70!black}}

\begin{document}

\begin{frame}
  \titlepage
\end{frame}

\begin{frame}{Oversikt}
	\begin{itemize}
		\item Praktisk Informasjon
		\begin{transparent}
		\item Gjennomgang av øving 7
		\item Programmering
		\end{transparent}
	\end{itemize}
\end{frame}

\begin{frame}{Oversikt}
	\begin{itemize}
		\begin{transparent}
		\item Praktisk Informasjon
		\end{transparent}
		\item Gjennomgang av øving 7
		\begin{transparent}
		\item Programmering
		\end{transparent}
	\end{itemize}
\end{frame}

\begin{frame}{Oversikt}
	\begin{itemize}
		\begin{transparent}
		\item Praktisk Informasjon
		\item Gjennomgang av øving 7
		\end{transparent}
		\item Programmering
	\end{itemize}
\end{frame}

% Alternatively, special title page command to get a different background
% \ntnutitlepage

\begin{frame}[fragile]{Rekursjon}
	\begin{columns}
		\begin{column}{.5\textwidth}
			\begin{itemize}
				\item En funksjon kaller seg selv med andre argumenter enn den ble kalt med selv
				\item Viktig at man har et case der funksjonen vil returnere uten å kalle seg selv, ellers vil vi fortsette i evig tid (eller til man stopper prosessen)
			\end{itemize}
		\end{column}
		\begin{column}{.5\textwidth}
			\begin{lstlisting}
def recursive(var):
	if <condition for termination>:
		return var # Can return something else, e.g. 1
	# Here, you can do something with var
	return var * recursive(var - 1)
			\end{lstlisting}
		\end{column}
	\end{columns}
\end{frame}

\begin{frame}[fragile]{Rekursjon eller iterativ løsning}
	\begin{columns}
		\begin{column}{.5\textwidth}
			\begin{itemize}
				\item \emph{Iterative} algoritmer bruker løkker for å løse et problem
				\item \emph{Rekursive} algoritmer bruker rekursjon
				\item Begge metodene kan brukes til å løse de samme problemene, men effektiviteten og kompleksiteten til løsningene vil ofte ikke være lik
			\end{itemize}
		\end{column}
		\begin{column}{.5\textwidth}
			\begin{lstlisting}
def recursive(n):
	if n == 0:
		return 1
	return n * recursive(n - 1)

def iterative(n):
	prod = 1
	for i in range(2, n+1):
		prod *= i
	return prod
			\end{lstlisting}
		\end{column}
	\end{columns}
\end{frame}

\begin{frame}{Oppgave 1}
	\begin{itemize}
		\item<+-> Binomialkoeffisienten til to tall $n$ og $k$ er gitt ved $\binom{n}{k} = \frac{n!}{k!(n-k)!}$, der $n! = n \cdot (n - 1) \cdot ... \cdot 1$. Lag en funksjon som tar inn to tall og finner binomialkoeffisienten deres.
		\begin{itemize}
			\item Hint: Lag en rekursiv funksjon som finner fakultetet av et tall
		\end{itemize}
	\end{itemize}
\end{frame}

\begin{frame}[fragile]{Oppgave 2}
	\begin{itemize}
		\item Lag en funksjon som tar inn en liste som parameter. Listen kan inneholde tall eller andre lister, som igjen kan inneholde tall eller andre lister, osv.
		\item Funksjonen skal returnere summen av alle tallene i listen, inkludert tallene i indre lister
		\begin{itemize}
			\item Eksempel: \lstinline|sum_list([1, 3, [5, 3], [6, [4, 2]]]) = 24|
			\item Hint: \lstinline|if isinstance(variable, list)|
		\end{itemize}
	\end{itemize}
\end{frame}

\begin{frame}{Oppgave 3}
	\begin{itemize}
		\item Lag en rekursiv funksjon som tar inn en liste og returnerer den reverserte lista. Du skal ikke bruke de innebygde \lstinline|reversed()|/\lstinline|lst.reverse()| funksjonene.
	\end{itemize}
\end{frame}

\begin{frame}{Oppgave 4}
	\begin{itemize}
		\item<+-> Lag en rekursiv funksjon som finner tverrsummen av et tall $n$, altså summen av sifferene i $n$
		\begin{itemize}
			\item Hint: Bruk \lstinline|n \% 10| og \lstinline|n // 10| for å finne og fjerne ett og ett siffer
		\end{itemize}
		\item<+-> Den minste tverrsummen til et tall $n$ er definert ved å ta tverrsummen, og hvis den har mer enn et siffer ta tverrsummen av det igjen osv. Lag en funksjon som finner den minste tverrsummen til et tall.
		\begin{itemize}
			\item Eksempel: Tverrsummen til $54321$ er $5 + 4 + 3+2+1=15$, og tverrsummen av $15$ er $1+5=6$, så den minste tverrsummen til $54321$ er $6$
		\end{itemize}
	\end{itemize}
\end{frame}

\begin{frame}[fragile]{Oppgave 5}
	\begin{itemize}
		\item<+-> Et firma har lagret informasjon om sine ansatte i en tekstfil. Fila består av et ukjent antall linjer på formen \lstinline|"<ansatt_id> <fornavn> <etternavn> <alder> <e-post> <lønn>"|, der ansatt\_id og alder er heltall, mens lønn er et desimaltall (anta at punktum brukes i desimaltall)
		\item<+-> Skriv en funksjon som leser inn en slik fil og returnerer en dictionary på formen \lstinline|{<ansatt_id>:[<fornavn>, <etternavn>, <alder>, <e-post>, <lønn>]}|
		\item<+-> Firmaet vil sende ut en julemail til alle sine ansatte, og vil personliggjøre mailen ved å bruke navnet til den ansatte. Skriv en funksjon som tar inn en slik dictionary, og returnerer en to-dimensjonal liste på formen \lstinline|[[<e-post tekst>, <e-post adresse>]]|. E-post teksten kan være det du vil, men må inneholde iallefall fornavnet på den ansatte.
	\end{itemize}
\end{frame}

\begin{frame}{Eksamen 2013, oppgave 4}
	\begin{itemize}
		\item<+-> \href{https://www.ntnu.no/wiki/display/tdt4110/Python+eksamensoppgaver?preview=/78972128/79298911/Python\%202013.pdf}{Ligger på wikien}
	\end{itemize}
\end{frame}

\begin{frame}{Spørsmål}
	\begin{itemize}
		\item Spørsmål kan også sendes til \href{mailto::olemagnp@stud.ntnu.no}{olemagnp@stud.ntnu.no}
	\end{itemize}
\end{frame}

\end{document}
